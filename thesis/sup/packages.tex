%
%   USED PACKAGES AND STYLES
%

%
%   KEINE AHNUNG; IST KRAM DEN JULIAN IN SEINEM TEMPLATE HATTE
%
\usepackage{newunicodechar}
\usepackage{comment}
%\usepackage{acro} %Abkürzungen
\usepackage{xcolor} %Farben
\usepackage{rotating}
\usepackage{lmodern}
%\usepackage{parskip}
%\usepackage[crop=off]{auto-pst-pdf} %Einbinden von .eps-Dateien in PDFTeX
\usepackage{paralist}

%
%   UTILITIES
%
%\usepackage{hyperref}

%
%   LANGUAGE, FONT
%
\usepackage[utf8]{inputenc} % Eingabe-Codierung
\usepackage[ngerman]{babel} % Spracheinstellung
\usepackage[T1]{fontenc}    % Anpassung auf europäischen Zeichensatz
\usepackage{xspace}         % Leerzeichen für Silbentrennung
\usepackage{hyphenat}       % Korrekte Silbentrennung


%
%   MULTIPLE FILES
%
\usepackage{pdfpages}   % Include pdf files
\usepackage{standalone} % Include .tex/.Rtex documents that have their own preamble and \begin{documents}

%
%   INHALTSVERZEICHNIS UND ÄHNLICHES
%
%%Umbennenung Inhaltsverzeichnis
\addto\captionsenglish{%
\renewcommand{\contentsname}{Inhaltsverzeichnis}}

%
%   FIGURES AND TABLES
%
\usepackage{subfig}                     % Mutli image figures
\usepackage[format=plain]{caption} %Einstellen der caption-Breite
\captionsetup[table]{skip=10pt}
\usepackage{wrapfig} %Floats von Text umflossen
\usepackage{float} %Position von figure/table erzwingen
\usepackage{flafter} %Floats nach Definition!
\usepackage{graphicx}
\usepackage{booktabs}
\usepackage{multicol}
\usepackage{multirow}


%
%   LITERATUR
%
\usepackage[style=chem-angew,
            doi=true,
			sorting=none,
			backend=biber,
			autocite=superscript,    % Zitate
			minbibnames=3,
			maxbibnames=5
			]{biblatex}
\usepackage{csquotes}                % Formattieren von Zitaten (Anfürhungszeichen)


%
%   MATHEMATIK
%
\usepackage{amsmath} %mathematischer Satz
\usepackage{amsfonts} %mathematische Schriftart
\usepackage{amssymb} %mathematische Symbole
\usepackage[output-decimal-marker={,},
            group-separator = {.}, 
            group-digits=true,
            detect-weight]{siunitx}     % korrekte Formatierung von Einheiten nach SI-Kriterien
\usepackage{nicefrac}
\usepackage{mathtools}  % pmatrix* environment for aligning matrix elements

%
%   CHEMIE
%
%\usepackage{chemgreek}
%\usepackage{chemnum} %Nummerierung der Verbindung aus ChemDraw
\usepackage[version=4]{mhchem} %Darstellung von chemischen Formeln
\usepackage{chemmacros}         % Abkürzungen (pH, pKa, ...) und Ladungen

%
%   FORMATTING, PAGE STYLE
%
\usepackage{setspace} %Zeilenabstand
%\usepackage{titlesec} %Verändern von Überschrift-Formatierungen
\usepackage[headsepline]{scrlayer-scrpage}

%
% INCLUDE SOURCE CODE
%
\definecolor{dkgreen}{rgb}{0,0.6,0}
\definecolor{gray}{rgb}{0.5,0.5,0.5}
\definecolor{mauve}{rgb}{0.58,0,0.82}
\usepackage{listings} %Programmiersprachen einbinden
\lstloadlanguages{R}
\lstset{frame=tb,
  language=R,
  aboveskip=3mm,
  belowskip=3mm,
  showstringspaces=false,
  columns=flexible,
  basicstyle={\small\ttfamily},
  numbers=none,
  numberstyle=\tiny\color{gray},
  keywordstyle=\color{blue},
  commentstyle=\color{dkgreen},
  stringstyle=\color{mauve},
  breaklines=true,
  breakatwhitespace=true,
  tabsize=4
}


